
\documentclass[12pt]{article} %make sure that we are using 12 pt font
\usepackage{fancyhdr}
\usepackage[margin=1.9cm]{geometry}%rounded up from 1.87, just to be safe...
\usepackage{parskip}
\usepackage{xcolor}
\usepackage{natbib}
\definecolor{darkblue}{RGB}{50,50,50}
\usepackage{hyperref}
\hypersetup{
	colorlinks=true, 
	linktocpage=true,
	breaklinks=true, 
	pageanchor=true,
	plainpages=false, 
	hypertexnames=true,
	urlcolor=darkblue, 
	linkcolor=darkblue,
	citecolor=darkblue, 
	pdftitle={Canada Tri-Council Grant},
	pdfauthor={Grant McKenzie},
	pdfsubject={Grant McKenzie},
	pdfkeywords={Grant McKenzie},
	pdfcreator={pdfLaTeX},
	pdfproducer={LaTeX via TeX Live on GNU/LINUX}
}
\usepackage{times} %make sure that the times new roman is used
\usepackage{titlesec}

% Set Section header sizes to normal font size
\titleformat*{\section}{\normalsize\bfseries}
\titleformat*{\subsection}{\normalsize\bfseries}

%\makeatletter

\fancyhf{}
\lfoot{\thepage} % page number in bottom left

\begin{document}
	
% Set top margin of page with relevant information
\pagestyle{fancy}
\rhead{Name, My}
\lhead{Name of Agency and Call -- Detailed Description}
\renewcommand{\headrulewidth}{0pt}

\raggedright

% I find it helpful to have the summary in the same document at the beginning.  Take it out before final PDF
%\pagestyle{empty}
\raggedright

\section*{Summary}


\pagebreak

% ======================== DETAILED DESCRIPTION ========================

\section{Objectives}


\section{Context}


\section{Methodology}



% ======================== LIST OF REFERENCES ========================
\pagebreak
\renewcommand{\section}[2]{} % Hide References title

% Reset Page Numbering
\clearpage
\pagenumbering{arabic}
 
% Reset Header for References
\lhead{Name of Agency and Call -- List of References}
\renewcommand{\headrulewidth}{0pt}


\bibliography{main.bib}
\bibliographystyle{apalike}

\end{document}
